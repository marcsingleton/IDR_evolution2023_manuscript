\begin{abstract}
\noindent
Here's where my abstract will go!
\end{abstract}

\section{Introduction}
Here's where my introduction will go!

\section{Results}
Here's where my results will go!

\section{Discussion}
Here's where my discussion will go!

\section{Materials and methods}

\subsection{Alignment and tree generation}

\subsection{IDR prediction and filtering}

\subsection{Fitting substitution models}

\subsection{Definition and calculation of features}
Features were calculated as in Zarin \textit{et al.} with the following modifications~\cite{Zarin2019}. The regular expression used for polar residue fraction was [QNSTCH], which, in contrast to the original study, excludes glycine residues. Additionally, length, expressed in log scale, was replaced with a feature proportional radius of gyration for an excluded-volume polymer~\cite{Flory1949}. Because the radii of gyration of chemically denatured proteins closely match the values expected for equivalent random coils, we felt this feature would better capture the relationship between an IDR's length and its biophysical properties~\cite{Kohn2004}. Finally, several motifs from ELM were replaced with their metazoan counterparts or updated versions of the same entries~\cite{Kumar2021}. These differences are noted in the supplementary data. Kappa, omega, SCD, hydropathy, PPII propensity, and Wootton-Federhen sequence complexity were calculated with localCIDER 0.1.19~\cite{Holehouse2017}. Isoelectric point was calculated with the Python package isoelectric, which is available on PyPI or at \url{https://isoelectric.org/}~\cite{Kozlowski2016}. Otherwise, features were calculated with custom code. A full list of features and their definitions is given in Table~\ref{stable:features} and Table~\ref{stable:regexes}.

\subsection{Brownian motion and Ornstein-Uhlenbeck analyses}

\subsection{Simulation analyses}

\subsection{GO term analyses}

\subsection{Code and data availability}
The code used to produce the results and analyses is available at \url{https://github.com/marcsingleton/IDR_evolution2023}. The following Python libraries were used: matplotlib, NumPy, pandas, SciPy, and scikit-learn~\cite{Hunter2007, Harris2020, McKinney2010, Virtanen2020, Pedregosa2011}. Relevant output files are available in the supporting information. There is no primary data associated with this manuscript. All primary data are available from publicly accessible sources described in their corresponding sections.
%======= OUTLINE =======
%
%BACKGROUND
%Survery of historical developments that inform dissertation
%Write at a high school or early college biology level
%
%I. Proteins are the molecular machines that drive biology
%    A. Four major macromolecules and briefly discuss broad roles of each
%        1. Nucleic acids
%        2. Carbohydrates
%        3. Lipids
%        4. Protein
%    B. Some major examples of proteins
%        1. Hemoglobin and sickle cell
%        2. Color blindness and opsins
%        3. Lactase and lactose intolerance
%II. Proteins are linear chains of amino acids with a three-dimensional structure
%    A. Background on biochemistry, primary, secondary, and tertiary structure
%    B. Structure-function relationships returning to previous hemoglobin example
%    C. Structural biology seeks to determine protein structures to understand function
%    D. Overview of structural methods including purification steps
%        1. X-ray crystallography
%        2. NMR
%        3. cryo-EM
%III. IDRs do not adopt a stable structure
%    A. History of disordered proteins and initial recognition
%        1. Success of structural methods
%    B. Disordered proteins are challenging to study with traditional methods
%    C. Genome sequencing and computational tools revealed disordered proteins are ubiquitous
%IV. IDRs have sequence-structure relationships (of a different kind)
%    A. Prompted by these observations, researchers began to clarify the sequence-structure-function relationships that characterize protein disorder
%        1. They discovered that related to structure and sequence but the details differ
%    B. Though structured proteins are not completely rigid, they generally have a small number of structures
%        1. As mentioned previously, even before disorder was recognized as a common feature of proteins, scientists understood proteins were not completely rigid
%        2. At the molecular level, everything is in constant motion. For example, at room temperature the average water molecule moves at a speed of SPEED.
%            i. However, collisions and interactions with other water molecules generally prevent them from escaping into the atmosphere
%        3. Proteins are also in motion, but they are constrained both by the rigid bonds and interactions between amino acid residues in the chain
%        4. Thus, all proteins flex and breathe. However, these motions are minor variations on the stable structure.
%        5. In other cases, proteins have multiple structures which are related to different functional states.
%            i. For example, hemoglobin has two forms, traditionally called T and R.
%            ii. The T form is hemoglobin's oxygen-free state.
%            iii. On binding oxygen, its structure changes slightly, with the shift on the scale of a few hydrogen atoms at the points of greatest change.
%            iv. This changes the orientation between the atoms in hemoglobin that interact with oxygen, allowing it to bind more tightly.
%            v. Promotes oxygen uptake at other three binding sites, promoting efficiently transport
%        6. Although proteins have multiple conformations, they are generally stable and the intermediates are not.
%        7. Thus, proteins with multiple conformations generally exist in discrete states.
%    C. Protein disorder means their atoms do not have fixed spatial relationships
%       1. In contrast, disordered proteins are described as ``conformational ensembles''
%       2. They are constantly sampling conformations which may have large differences in the spatial relationships between residues
%       3. The conformations of disordered regions may adopt more extended conformations or collapsed conformations
%       3. The composition of a disordered region dictates its conformational profile. However, to describe
%       4. As mentioned previously, proteins are linear chains of amino acids, and the amino alphabet contains 20 letters
%       5. Though each amino acid is chemically unique, many of their properties can be understood in terms of
%       1. Disordered regions largely characterized by enrichment and depletion in hydrophilic and hydrophobic residues, respectively
%       2. Hydrophobic residues are ``water-fearing,'' and like oil in water, they tend to self-associate.
%       3. Hydrophilic residues are ``water-loving,'' and easily interact with water.
%       4. Hydrophobic interacts are a major force driving protein folding, and many folded proteins have a hydrophobic core
%    E. Sequence-structure relationships in disordered proteins
%       1. Although, disordered regions are conformationally heterogeneous, they
%V. IDRs are functional
%    A. Flexibility confers several advantages
%        1. The structural flexibility of IDRs is directly related their functional plasticity
%        2. Proteins involved in signaling, transport, and catalysis require specific geometries to interact their intended targets efficiently
%        3. Disordered proteins can interact with multiple targets
%            i. Environmental signals can fine-tune which interacts are preferred
%            ii. Proteins, structured and disordered, frequently undergo modifications which change their behavior
%            iii. A common modification is phosphorylation where a phosphate group is attached to a specific amino acid in the protein
%                a. Phosphate groups contain three negative charges, so the addition can dramatically affect a protein's folding energetics and induce a change in structure
%            iv. Thus, phosphorylations commonly toggle proteins between an inactive and active state
%                a. Proteins are phosphorylated by a class of proteins called kinases, which are in turn activated by interactions with or modifications by other proteins
%                b. A single stimulus can set off a sequence of events, called a signaling cascade, spanning hundreds of reactions between as many distinct proteins
%                c. Biologists are sometimes descriptive to a fault, resulting in proteins like MAP kinase kinase kinase which phosphorylates MAP kinase kinase which phosphorylates MAP kinase which phosphorylates the MAP protein. [FOOTNOTE]
%            v. Because IDRs expose much of their chain to the surrounding environment, they have many sites which are available for modification
%                a. Thus, they often function as hubs in networks of signaling pathways [CITATION]
%                b. Some IDRs are roughly like molecular computers which integrate signals from varying sources
%    B. Survey of disordered proteins and their functions
%        1. Linkers
%        2. Signaling
%VI. IDRs are enriched in transcriptional machinery
%    A. Introduce transcription via the central dogma
%    B. Describe transcription factors, cofactors, transcriptional machinery
%        1. Examples of different transcription factors
%    C. Describe properties of transcription factors
%        1. DNA binding and activation domains
%        2. Different flavors of activation domains
%    D. How transcription factors find their targets to initiate specific genetic programs is one of the major outstanding questions in molecular biology
%VII. IDRs are associated with phase separation
%    A. Describe phase separation in general
%    B. Describe phase separation as related to proteins
%    C. Describe functions of phase separation and relationship to subcellular structures
%VIII. IDR-mediated phase separation is a possible mechanism for initiating transcription
%    A. Summarize Stadler's papers and others
%
%AIMS
%I. Despite significant advances in predicting disorder from sequence alone and relating those predictions to specific conformational profiles, the relationship between sequence and function in IDRs remains poorly understood. Thus, functions are not readily associated with arbitrary IDRs.
%    A. In contrast, folded domains can identified by characteristic sequences of amino acids
%        1. Because structured domains make specific contacts between residues, their sequences of amino acids are generally conserved or evolve slowly
%        2. Thus, for structured domains their specific sequence of amino acids constitute a unique signature
%        3. Traditional bioinformatics methods use these signatures to detect similarities between proteins
%        4. Structural and functional information is transferred between proteins with similar sequences
%        5. In practice, however, this is extremely difficult. Fully mapping the relationship between sequence and structure of folded domains was an active area of research for decades
%            a. Recent developments in machine learning have in many senses solved this problem, but still required an advanced computational model trained on nearly two hundred thousand experimentally determined structures.
%    B. Disordered proteins, however, challenge this sequence-dependent model of protein structure and function.
%    X. Though there are exceptions, many IDRs evolve extremely rapidly and are not identifiable by the specific sequence of amino acids
%    C. However, growing evidence that IDRs evolve under a different set of constraints
%        0. Rather than preserving a specific sequence in order to preserve contacts and structure, they preserve distributed features such as composition, length, charge, or the presence of motifs
%        1. Because many sequences can yield an IDR with a specific conformational profile or local arrangements, they are not constrained to preserve the precise order of amino acid residues
%    D. Examples of studies that have found specific cases of conservation in IDRs
%        1. Earlier specific examples of feature conservation in proteins
%        2. However, a recent study from Zarin et al was the first to demonstrate of widespread of feature conservation in IDRs across an entire proteome, that is the set of all proteins found in an organism
%    E. The analyses used simulations and evolutionary comparisons across a set of yeast genomes, a foundational model organism in molecular biology research. However, no subsequent studies have investigated similar patterns of conservations, or any patterns of conservation at all, characterize in IDRs in other systems.
%        0. No conservation global patterns of conservation would suggest such preserved conservation are unique to yeast
%        1. Patterns of conservation but are distinct from yeast would indicate global patterns
%        2. Similar patterns of conservation with similar functional associations would indicate a universal taxonomy of IDRs
%II. Use evolutionary analyses to identify conserved properties of disordered regions
%    A. Though modern molecular genetic and gene editing techniques it is in principle possible to test the relationship between distributed features and the structure and function of specific proteins in living systems directly, the vast number of combinations make this approach infeasible in practice.
%    B. Instead life constantly explores the space of allowed proteins through evolutionary changes
%    C. Properties which are unimportant for a protein's function, or at a larger scale, an organism's survival offer no benefit for their maintenance and are gradually degraded and ultimately lost
%    A. Thus, conservation is powerful signal of function, which are illuminated by evolutionary comparisons between related sequences.
%    X. Such comparisons, however, require the identification of IDRs with common ancestry that perform ``equivalent'' functions across many distinct organisms. Given the difficulties with identifying equivalent IDRs, this may seem like a chicken and egg problem. Fortunately IDRs are frequently associated with more conserved folded domains. Thus, identifying evolutionarily related sequences by their overall sequence signatures and ``aligning'' can in turn identify IDRs with common ancestry. Thus the first step of these analyses is the identification of such proteins with common ancestry, called orthologs. Since the rise full genome sequencing in the late 1990s, researchers have developed computational methods for identifying orthologs and aligning them. While these methods are generally effective, they are conducted by automated computational pipelines and therefore prone to errors, especially when highly divergent disordered. Fortunately over the past five years, advances in DNA sequencing technology has yielded a dramatic increase in the number of genomes available for analysis. However, conventional methods have not fully leveraged this increase in information to improve the process of identifying and aligning orthologous sequences. Thus, in my first chapter I developed new methods for leveraging this information and apply it them to a set of Drosophila genomes. Drosophila is a
% In my second chapter, I use the alignments generated in the first chapter to analyze IDRs with a variety of evolutionary models. Finally, in my third chapter I discuss several general purpose computational tools I developed in the process.
% X. Finally in the third chapter
%======= DRAFT =======

\begin{abstract}
\noindent
Here's where my abstract will go!
\end{abstract}

\section{Background}
Life is a physical phenomenon. Despite the complexity of living things, their processes are governed by the same physical laws that describe planets' motion around the sun and the propagation of electromagnetic waves through space. However, many systems are too complex to describe with physical models and equations, so scientists simplify them into levels of abstraction that are more useful. (This is the origin of the common observation that biology is applied chemistry and chemistry is applied physics.) For example, Punnett squares facilitate the prediction of genotypes and phenotypes by distilling the complexities and nuances of diverse reproductive systems into a set of simple rules. However, since life spans a scale from single cells to entire ecosystems, biology likely employs more layers of abstraction than any other scientific discipline. One of the most powerful and widely used frameworks within the life sciences is biochemistry, which characterizes biological processes in terms of their component molecules and chemical reactions. A specific focus is four classes of macromolecules called nucleic acids, carbohydrates, lipids, and proteins, all of which are unique to biological systems. Though not all biological molecules are macromolecules\footnote{Macromolecule is a loose term applied to molecules with high molecular mass. In practice, it usually refers to one of the classes listed above, among a few other prominent non-biological examples.} and not all biological macromolecules fit neatly in one of these four categories, much of life at the molecular level is understood in terms of their structure and function. Each of the four has a characteristic role. Nucleic acids, \textit{i.e.} DNA and RNA, are responsible for information storage and transfer. Carbohydrates primarily store energy but can also act as structural components of cells. Lipids are a diverse class of oily molecules which are components of cell membranes, store energy, and transmit signals. Proteins have a range of functions, including catalyzing reactions, transmitting signals, transporting materials, and providing structure. Many of these overlap with the functions of the other macromolecule classes because proteins are involved in virtually every biological process. However, unlike the others, which are often passive participants, proteins are highly active and dynamic. They respond to signals, change shape, and often use the other macromolecules as substrates in their activities. Proteins are essentially the molecular machines that carry out life's functions.

Some examples will illustrate the central role of proteins more clearly. Blood is a part of the circulatory system which is responsible for transporting nutrients and waste. Though blood is a complex mixture, containing a cocktail of cells, proteins, sugars, gases, and ions dissolved in a medium of water, its primary cellular component is red blood cells. These cells, which give blood its red color, ferry oxygen from lungs throughout the body. While water can dissolve some oxygen, the body requires more oxygen more quickly than is available in the aqueous component of blood alone. Thus, red blood are packed with a special protein called hemoglobin, which binds\footnote{In biology, ``binding'' is a slippery word whose exact meaning can varying greatly depending on the context. However, it generally means that a molecule physically interacts with another molecular for an extended period.} oxygen. Each red blood cell contains as many as 270 million molecules of hemoglobin, each of which can carry up to four oxygen molecules~\cite{Pierig2008}. Because red blood cells are so dense with hemoglobin, composing roughly 35\% of their total volume, any defect in hemoglobin can dramatically impact the structure of the red blood cells themselves~\cite{Kanias2009}. A well-studied example is sickle cell disease where an error in the body's hemoglobin molecules deforms red blood cells into a characteristic sickle shape. This prevents them from easily flowing through blood vessels, resulting in pain and oxygen deprivation.

Whereas hemoglobin is an example of a protein mediating transport, proteins are also involved in transmitting signals and catalyzing chemical reactions. For example, the back of the eye contains a light-sensitive surface called the retina which is composed of photoreceptor cells. These cells respond to light because they produce special proteins called opsins that translate light into chemical and electrical signals which are then interpreted by the brain. Humans, and primates broadly, have three types of opsins that mediate our color vision, which are sensitive to red, green, and blue light, respectively. Color blindness is the result of photoreceptor cells missing one of these proteins, typically either the red or green opsin. In contrast to sickle cell disease, this condition is caused by a missing rather than a mutated protein. However, in other cases, a protein is not missing or mutated, but instead not produced at the right time and place. For example, lactose is a sugar found in milk that requires a specific protein, lactase, to metabolize properly. Many humans who can digest milk products in childhood, lose this ability in adulthood because they stop producing lactose. As a result, lactose in dairy products passes undigested into the colon where it is broken down by bacteria, causing symptoms such as bloating and diarrhea.

Despite performing this diverse range of functions, all proteins are made from of a set of 20 simple building blocks called amino acids.\footnote{The term amino acid encompasses any compound that contains an amino and carboxyl group. However, proteins are only synthesized from the 20 ``canonical'' amino acids. Another two (selenocysteine and pyrrolysine) are incorporated via a distinct mechanism under rare circumstances and are therefore considered non-standard.} Though each amino acid is chemically unique, they share a common backbone composed of two distinct and complementary receptor and donor sites for chemical bonds. Thus, in a protein the amino acids are bonded in a linear chain like beads on a string. However, once synthesized, proteins are not tidy rod-shaped molecules. Instead, the chain loops and weaves between itself creating a three-dimensional structure in a process called folding. These structures, which are highly stable and characteristic of each protein, are a result of the interactions between the amino acids in the chain and the surrounding medium, which is typically water. Because each amino acid has unique geometric and chemical properties that influence the energetics of these interactions, a protein's three-dimensional structure is encoded by the sequence of amino acids that compose it. A protein's function is in turn a direct result of its structure. For example, the structure of hemoglobin precisely positions its amino acids and a helper molecule called a heme group to create a pocket that can stably but reversibly bind oxygen. This allows hemoglobin to carry oxygen throughout the body until it is delivered to its destination. However, people affected by sickle cell disease have a mutation in the sequence of their hemoglobin proteins which causes it to malfunction. Frequently this mutation is a single change where the sixth amino acid in the sequence, glutamate, is substituted for a valine. This creates a sticky patch on the surface of hemoglobin, and under low-oxygen conditions normal hemoglobin changes shape to expose a sticky patch on its surface as well. The two patches are complementary, which allows hemoglobin proteins to clump together into long, fibrous strands. These strands distort the shape of red cells, giving them their characteristic sickle shape.

Clearly, understanding the relationship between the sequence, structure, and function of proteins is essential for unraveling more complex biological phenomena. Though biologists study all three properties of proteins, they are generally most interested in function since it is the most directly related to the biological processes the protein takes part in\footnote{Function generally refers to \textit{molecular} function which is a description of a specific chemical activity possessed by a protein. A biological process, however, is the larger ``biological program'' which is accomplished by the action of multiple linked molecular functions. For example, the molecular function of hemoglobin is to bind oxygen, an activity it shares with a related protein myoglobin. However, the two have different roles in the process of oxygen transport and storage. Hemoglobin is found in red blood cells where it acts as a carrier during transport. In contrast, myoglobin is found in muscle cells, where it stores oxygen until needed.}, function and biological process are not always easily identified or measured. Thus, determining a protein's structure is frequently the first step of detailed studies of its function. Though in recent years researchers have developed powerful computational tools that can accurately predict structure from sequence alone, historically structures were determined experimentally, and experimental methods still remain the gold standard. While many methods can reveal information about the structure of a protein, the most powerful techniques, X-ray crystallography, NMR spectroscopy, and cryogenic electron microscopy (cryo-EM), can map the spatial coordinates of every atom in a protein. However, this resolution requires extremely pure samples of the protein of interest. Since proteins are only produced by living systems, preparations begin with a complex mixture consisting of cells or tissue, and the protein of interest is isolated with series of extraction and purification steps. Some proteins are only produced in small amounts or degrade easily, so each step may require substantial optimization to achieve a sufficient yield. When structural techniques were first developed in the late 1950s, they were so time-consuming that a graduate student could dedicate an entire PhD to solving a single protein structure. Many developments have substantially accelerated the process, but it remains a labor-intensive technique which may require several months of effort. However, the result is a powerful map that scientists use to suggest hypotheses and interpret data.

The success of structural methods at elucidating the molecular details of protein function cemented the view that function largely depends on the presence of a fixed structure. While scientists understood proteins were not completely rigid and could adopt a variety of related structures, many believed that functional proteins largely had a single dominant structure~\cite{Karush1950}. Despite its strength, exceptions to the structure-function paradigm were known. For example, elastin is a protein secreted by cells which allows tissues like skin or blood vessels to repeatedly expand and contract. It imparts this elasticity by forming networks of disordered chains that act like molecular springs. When a tissue experiences a force, the chains stretch to accommodate it. When the force is removed, the chains return to their random orientations, which reduces their end-to-end length and forces the tissue to return to its original shape~\cite{Vrhovski1998, Alberts2014}. However, as a result of this unique role in providing tissue elasticity, elastin's disorder was viewed as a specific adaptation rather than a general mechanism of protein function. In other cases, proteins had regions which returned undefined or highly variable atomic coordinates when analysed with structural techniques, indicating they lacked defined structures and were ``disordered.'' Because these segments were often short loops between structured regions, they were seen as ``linkers'' which facilitated the structure of the functional portions of proteins. By the early 2000s, though, enough exceptions had accumulated that scientists began to recognize that fully and partially disordered proteins were involved in many biological processes~\cite{Plaxco1997, Wright1999, Dunker2001}. Many of these examples were proteins which folded on binding to their targets, commonly other proteins. This mechanism was a departure from the prevailing model of interactions between biological macromolecules which required highly stable and complementary interfaces, like two puzzle pieces fitting together. As a result, scientists speculated that disorder was an adaptation that allowed proteins to efficiently relay and regulate signals by enabling interactions with many possible targets. Furthermore, the flexibility of disordered protein would permit environmental conditions to easily modulate these interactions.

In the following years, as the complete genomes of several scientifically important model organisms such as \textit{S. cerevisiae} (baker's yeast), \textit{C. elegans} (roundworm), and \textit{D. melanogaster} (fruit fly) were sequenced for the first time, researchers applied computational methods for predicting disorder to the proteins inferred from their genomes. They discovered that disorder is ubiquitous in eukaryotic\footnote{All life belongs to one of three categories, or domains. Two, Archaea and Bacteria, are all single-celled organisms with simple cellular structures. In contrast, the cells of members of Eukarya, \textit{i.e.} eukaryotes, are complex and contain substructures called organelles, among other differences. All animals, plants, and fungi are eukaryotes, though the domain includes many microorganisms as well.} organisms, with estimates of the fraction of proteins containing disordered segments of greater than 30 residues\footnote{The amino acids that compose the links of a protein chain are conventionally called residues to distinguish them from their related, but chemically distinct, free forms.} ranging between 28 and 63\%~\cite{Dunker2000, Ward2004}. For reference, though the lengths of proteins can vary dramatically, a ``typical'' protein contains on the order of a few hundred residues, so these regions can compose a significant fraction of a protein's length. Because these segments were disordered in their native state, \textit{i.e.} under normal operating conditions, such segments were termed ``intrinsically disordered regions'' (IDRs) to emphasize the disorder was not induced by exposure to chemicals or heat. Furthermore, while most proteins were predicted to contain a mixture of structured domains and IDRs, some ``intrinsically disordered proteins'' (IDPs) were entirely or almost entirely disordered. Thus, disorder was recognized as a pervasive but poorly understood feature of proteins.

Many studies investigated the structural and functional properties of IDRs over the following years. They found that although IDRs still have sequence-structure-function relationships, they play by a very different set of rules. These differences manifest at all three levels but are at first most easily understood in terms of structure. Strictly speaking, a protein's structure refers its three-dimensional arrangement of atoms. Thus, ``structured'' regions in proteins, often called domains\footnote{Though there are various overlapping definitions, domains typically refer to independently folding regions of a protein. Domains are also described as discrete functional or evolutionary elements since proteins may contain several domains which are connected by unstructured ``linker'' sequences.}, typically fold into a small number of related structures. This does not imply these folded domains are completely rigid, though. At the molecular level, everything is in constant motion. For example, at room temperature an average water molecule moves at over 500 meters per second!\footnote{This value was derived using the Maxwell–Boltzmann distribution, which is a physical model of the speeds of particles in an ideal gas. Clearly, water is not a gas at room temperature, so it should be considered a rough approximation.} However, liquid water is so dense that it will collide with something after moving only a fraction of its own length. Likewise, in the cellular environment folded domains are buffeted by collisions with water and other molecules, but they are constrained by the rigid bonds and interactions between amino acid residues in the chain. Thus, while folded domains can ``flex'' and ``breathe,'' these motions are minor variations on their overall structure.

In some cases, folded domains have multiple structures, or ``conformations,'' which are related to different functional states. For example, hemoglobin has two forms, traditionally called the T and R states. The T state is hemoglobin's oxygen-free form, but on binding oxygen its structure shifts to the R state. The change is small, differing at most by only a few hydrogen atoms. However, this enough to re-orient the atoms that interact with oxygen, allowing it to bind more tightly and promoting oxygen uptake at the other three binding sites. Once the red blood cells reach their destination, other physiological factors favor the adoption of the T state, which coordinates the release of all four oxygen atoms. In the other cases, conformational changes can dramatically re-organize a protein's structure. For example, the 26S proteasome is a complex of proteins responsible for degrading other proteins. Its structure is highly complex and consists of over three dozen distinct protein ``subunits,'' which are in turn organized into three subcomplexes: a lid, a base, and a core~\cite{Finley2016, Bard2018}. The functions of these subcomplexes are roughly analogous to the parts of a paper shredder. The lid is like the outer shell because it regulates access to the ``motor'' in the base and the ``blades'' in the core that pull in and degrade the protein, respectively.\footnote{As with many analogies, this comparison to a paper shredder simplifies several structural and functional details of the proteasome. For example, in a paper shredder the motor powers the blades which both pull in and shred the paper. In the proteasome, however, these are distinct steps. The motor subunits in the base first physically interact with the protein to simultaneously unfold and pull it into the base. A different set of subunits in the base then break the exposed chemical bonds between amino acid residues in the chain.} Like a paper shredder, the motor is only engaged when a protein is correctly positioned in the lid. Unlike a paper shredder, however, the motor is activated by a conformational change rather than a physical switch. When the tail of a protein marked for degradation is inserted in the motor, the lid shifts by nearly forty hydrogen atoms to align the motor with the pore that leads into the core subcomplex. Despite the scale of this re-arrangement, it occurs over the span of only half a second~\cite{Bard2019}. Thus, when folded domains have multiple conformations, they are generally discrete forms without stable intermediates.

In contrast, IDRs have no fixed spatial relationship between their atoms, so their structures are sometimes described as ``conformational ensembles,'' \textit{i.e.} collections of conformations where the distances and orientations between residues can vary considerably. Furthermore, IDRs populate a continuum of structural states over time whereas when folded domains undergo dramatic conformational changes, they are usually triggered by specific environmental signals or chemical modifications, and the intermediate structures are transitory. Despite the diversity of conformations available to IDRs, they can be broadly grouped into one of several qualitative descriptions which range from extended ``coils'' to more compact ``globules.'' The specific conformational class of a given IDR, however, is dictated by its local composition of amino acid residues. Because the chemical properties of each amino acid in the protein alphabet are dictated its specific arrangement of atoms, each has a unique impact on a protein's structure. However, to simplify discussion and analysis amino acids are often compared by quantitative factors like size or charge. One of the most useful scales for describing an amino acid's overall effect on protein structure is hydrophobicity, which measures a molecule's tendency to associate with water. Molecules which attract water are hydrophilic (``water loving''), and molecules which repel water are hydrophobic (``water fearing''). Hydrophilic molecules, like sugar or alcohol, easily dissolve in water whereas hydrophobic molecules, like fats and oils, remain separate. Since the cellular environment is largely water, the hydrophobic residues in proteins tend to aggregate into a ``hydrophobic core.'' This ``hydrophobic collapse'' is a major driving force in the early stages of protein folding, so a protein's relative number of hydrophilic and hydrophobic residues is a key determinant of whether it is folded or disordered.

Unsurprisingly, disordered regions are characterized by a relative depletion of hydrophobic residues. However, there is no simple formula which accurately predicts disorder in a protein using only the hydrophobicity values of its constituent residues. Other factors, such as the number and distribution of charged residues, also impact a sequence's predisposition for disorder~\cite{vanderLee2014, Das2015}. For example, sequences with high number of either positively or negatively charged residues, called polyelectrolytes, tend to form stiff rods because the like charges repel each other. However, if a sequence contains a high number of positively and negatively charged residues in roughly equal proportion, the sequence is called a strong polyampholyte, and its conformational class depends on the distribution of those charged residues. If the two classes of residues are segregated into separate blocks of like charges, they attract and form hairpins. However, if they are evenly distributed, the attractive and repulsive forces balance on average, and the sequence generally assumes expanded coil-like conformations. High numbers of polar amino acids, which are hydrophilic but not charged, are associated with semi-compact globules. Though their ``side chains,'' the portions which give each amino acid its unique identity, are hydrophilic, their interactions with water are not sufficient to overcome the tendency of the hydrophobic ``backbone,'' composed the donor and receptor sites common to all amino acids, to self-associate. Thus, like folded domains, the structures of disordered proteins are dictated by their sequences. However, because IDRs do not make stable contacts between specific residues in their chains, multiple sequences can generally correspond to a single conformational class.

The structural diversity of IDRs is directly related to their functional plasticity.

Although the previous discussion has presented of disorder and structure as distinct phenomona , disorder is a spectrum
MorFs
folding on binding
fuzzy complexes

\section{Aims}

Despite recent advances in identifying IDRs and their conformational ensembles from their sequences alone, the relationship between the sequence and function remains poorly understood. In contrast, predictions of structure and function are readily available for many folded domains. Because they make specific contacts between residues, the sequences of folded domains are generally conserved or evolve slowly. Thus, the specific sequence of a folded domain constitutes its unique ``signature.'' Traditional bioinformatics techniques use these signatures to detect similar protein sequences and transfer structural and functional information between them~\cite{Camacho2009, Eddy2009, Mistry2020}. Though this approach is simple in principle, it is extremely difficult to perfect in practice, and fully mapping the relationship between the sequence and structure of folded domains alone was an active area of research for decades. Part of the challenge is the available data is extremely sparse relative to the sheer number of possible protein sequences. Even for a moderately sized protein of 100 amino acid residues, there are $20^{100} \approx 1.3 \times 10^{130}$ possible sequences. Only in the past few years have researchers in many senses solved this problem by using advanced techniques from machine learning to leverage the information encoded in nearly two hundred thousand experimentally determined structures~\cite{Jumper2021}.

IDRs, however, challenge this sequence-dependent model of protein structure and function. Because they do not make stable contacts between residues which establish a fixed structure, IDRs are not generally constrained to maintain a specific sequence of residues. Thus, while there are exceptions, many IDRs evolve extremely rapidly, and related IDRs are therefore not easily identified by their sequence. There is growing evidence, though, that IDRs evolve under a different set of constraints. Because the composition and patterning of residues in an IDR dictates its conformational class, many distinct sequences can yield a similar set of conformational ensembles. Furthermore, because binding motifs and modification sites in IDRs are usually fewer than ten residues, their interaction interfaces are compact and can occur in multiple positions without compromising function~\cite{Tompa2014}. Thus, rather than conserving specific sequences, IDRs conserve distributed ``molecular features'' associated with those sequences. By the mid 2010s several studies had demonstrated evidence of such constraint in the flexibility, chemical composition, net charge, or charge distribution of IDRs~\cite{Daughdrill2007, Moesa2012, Zarin2017, Beh2012}. While these earlier studies were generally restricted to specific features or proteins, in 2019 Zarin \textit{et al.} demonstrated conservation of various features in a comprehensive set of IDR-associated properties across IDRs in the entire yeast proteome~\cite{Zarin2019}. By comparing the observed values of these features against those generated under a simulated model of evolution, they clustered IDRs by their ``evolutionary signatures,'' \texit{i.e.} patterns of conserved features. Furthermore, they identified specific biological functions associated with these groups, which for the first time provided a global view of the relationship between sequence and function in IDRs.

These analyses were conducted using a set of IDRs identified in various species of yeast, which is a widely used model organism in molecular biology research. However, no known subsequent studies have determined if similar patterns of conservation are found in the IDRs of other systems. As another foundational model organism with abundant genomic information across many evolutionary lineages, the fruit fly, \textit{Drosophila melanogaster}, is a natural choice for subsequent investigation~\cite{Yang2018, Miller2018, Kim2021}. Furthermore, given its complex multicellular development process and shared signaling pathways with humans, the findings of such a study would significantly advance our understanding of the role of IDRs in human health and disease. The concordance of these results with the previously identified IDR clusters would also have profound implications for the broader mechanisms of IDR evolution. For example, the absence of global patterns of evolutionary signatures across IDRs in \textit{Drosophila} would suggest they are property of IDRs which is unique to yeast. In contrast, the identification of clusters similar to those in yeast would indicate the existence of a taxonomy of IDRs which is conserved across the tree of life.

Though modern genetic engineering techniques enable the direct manipulation of DNA sequences in living systems, gene editing remains a lengthy and work-intensive process in fruit flies. Therefore, experimentally testing the vast number of sequences needed to fully map the relationship between distributed features and function in IDRs is infeasible. The comparative genomics approach instead leverages the work done by nature to identify evolutionarily conserved features and generate specific hypotheses to guide experiments~\cite{Hardison2003}. As life is constantly exploring the space of allowed proteins through evolutionary change, features which are unimportant for a protein's function, or at a larger scale, an organism's survival offer no benefit for their maintenance and are therefore gradually degraded and lost. Thus, conservation in related sequences is powerful signal of function.

These comparisons, however, require the identification of IDRs with common ancestry that perform ``equivalent'' functions across many distinct organisms. Given the difficulties with identifying similar IDRs by their sequences, this may seem like a chicken and egg problem. Fortunately, IDRs are frequently associated with more conserved folded domains. Thus, identifying evolutionarily related proteins by their overall sequence signatures and ``aligning'' them will in turn identify the equivalent IDRs in those sequences. The first step of an evolutionary analysis of IDRs, then, is the identification of proteins with common ancestry, called orthologs. Since the first genomes were sequenced in the late 1990s, researchers have developed techniques for identifying and aligning orthologs~\cite{Fleischmann1995, Goffeau1996, CESC1998, Tatusov1997}. While these methods are generally effective, they are conducted by automated computational pipelines and prone to errors when processing the highly divergent sequences that characterize many IDRs. The evolutionary relationships between the genomes of closely related species generally make such mistakes easier to identify, and fortunately over the past five years advances in DNA sequencing technology have yielded dramatic increases in the number of sequenced genomes in the \textit{Drosophila} genus. However, because the existing methods for ortholog identification were designed for fewer or more distantly related genomes, they do not fully leveraged the available genomic redundancy to minimize errors. Thus, in the first chapter I develop a novel method for identifying orthologs which addresses this shortcoming and apply it to 33 \textit{Drosophila} genomes to generate a set of aligned orthologs. In the second chapter I then identify rapidly evolving IDRs in these ``alignments'' and analyse them with a variety of evolutionary models to detect patterns of conservation. Finally, in the third chapter I discuss several software tools and tutorials for fitting statistical models to data, which were created while pursuing the previous aims.

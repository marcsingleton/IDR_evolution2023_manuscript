%======= OUTLINE =======
%
%BACKGROUND
%Survery of historical developments that inform dissertation
%Write at a high school or early college biology level
%
%I. Proteins are the molecular machines that drive biology
%    A. Four major macromolecules and briefly discuss broad roles of each
%        1. Nucleic acids
%        2. Carbohydrates
%        3. Lipids
%        4. Protein
%    B. Some major examples of proteins
%        1. Hemoglobin and sickle cell
%        2. Color blindness and opsins
%        3. Lactase and lactose intolerance
%II. Proteins are linear chains of amino acids with a three-dimensional structure
%    A. Background on biochemistry, primary, secondary, and tertiary structure
%    B. Structure-function relationships returning to previous hemoglobin example
%    C. Structural biology seeks to determine protein structures to understand function
%    D. Overview of structural methods including purification steps
%        1. X-ray crystallography
%        2. NMR
%        3. cryo-EM
%III. IDRs do not adopt a stable structure
%    A. History of disordered proteins and initial recognition
%        1. Success of structural methods
%    B. Disordered proteins are challenging to study with traditional methods
%    C. Genome sequencing and computational tools revealed disordered proteins are ubiquitous
%IV. IDRs have sequence-structure relationships (of a different kind)
%    A. Prompted by these observations, researchers began to clarify the sequence-structure-function relationships that characterize protein disorder
%        1. They discovered that related to structure and sequence but the details differ
%    B. Though structured proteins are not completely rigid, they generally have a small number of structures
%        1. As mentioned previously, even before disorder was recognized as a common feature of proteins, scientists understood proteins were not completely rigid
%        2. At the molecular level, everything is in constant motion. For example, at room temperature the average water molecule moves at a speed of SPEED.
%            i. However, collisions and interactions with other water molecules generally prevent them from escaping into the atmosphere
%        3. Proteins are also in motion, but they are constrained both by the rigid bonds and interactions between amino acid residues in the chain
%        4. Thus, all proteins flex and breathe. However, these motions are minor variations on the stable structure.
%        5. In other cases, proteins have multiple structures which are related to different functional states.
%            i. For example, hemoglobin has two forms, traditionally called T and R.
%            ii. The T form is hemoglobin's oxygen-free state.
%            iii. On binding oxygen, its structure changes slightly, with the shift on the scale of a few hydrogen atoms at the points of greatest change.
%            iv. This changes the orientation between the atoms in hemoglobin that interact with oxygen, allowing it to bind more tightly.
%            v. Promotes oxygen uptake at other three binding sites, promoting efficiently transport
%        6. Although proteins have multiple conformations, they are generally stable and the intermediates are not.
%        7. Thus, proteins with multiple conformations generally exist in discrete states.
%    C. Protein disorder means their atoms do not have fixed spatial relationships
%       1. In contrast, disordered proteins are described as ``conformational ensembles''
%       2. They are constantly sampling conformations which may have large differences in the spatial relationships between residues
%    D. Protein folding driven by hydrophobic interactions
%       1. Disordered regions largely characterized by enrichment and depletion in hydrophilic and hydrophobic residues, respectively
%       2. Hydrophobic residues are ``water-fearing,'' and like oil in water, they tend to self-associate.
%       3. Hydrophilic residues are ``water-loving,'' and easily interact with water.
%       4. Hydrophobic interacts are a major force driving protein folding, and many folded proteins have a hydrophobic core
%    E. Sequence-structure relationships in disordered proteins
%       1. Although, disordered regions are conformationally heterogeneous, they
%V. IDRs are functional
%    A. Flexibility confers several advantages
%        1. Structural flexibility is directly related to functional plasticity
%        2. Proteins involved in signaling, transport, and catalysis require specific geometries to interact their intended targets efficiently
%        3. Disordered proteins can interact with multiple targets
%            i. Environmental signals can fine-tune which interacts are preferred
%            ii. Proteins, structured and disordered, frequently undergo modifications which change their behavior
%            iii. A common modification is phosphorylation where a phosphate group is attached to a specific amino acid in the protein
%                a. Phosphate groups contain three negative charges, so the addition can dramatically affect a protein's folding energetics and induce a change in structure
%            iv. Thus, phosphorylations commonly toggle proteins between an inactive and active state
%                a. Proteins are phosphorylated by a class of proteins called kinases, which are in turn activated by interactions with or modifications by other proteins
%                b. A single stimulus can set off a sequence of events, called a signaling cascade, spanning hundreds of reactions between as many distinct proteins
%                c. Biologists are sometimes descriptive to a fault, resulting in proteins like MAP kinase kinase kinase which phosphorylates MAP kinase kinase which phosphorylates MAP kinase which phosphorylates the MAP protein. [FOOTNOTE]
%            v. Because IDRs expose much of their chain to the surrounding environment, they have many sites which are available for modification
%                a. Thus, they often function as hubs in networks of signaling pathways [CITATION]
%                b. Some IDRs are roughly like molecular computers which integrate signals from varying sources
%    B. Survey of disordered proteins and their functions
%        1. Linkers
%        2. Signaling
%VI. IDRs are enriched in transcriptional machinery
%    A. Introduce transcription via the central dogma
%    B. Describe transcription factors, cofactors, transcriptional machinery
%        1. Examples of different transcription factors
%    C. Describe properties of transcription factors
%        1. DNA binding and activation domains
%        2. Different flavors of activation domains
%    D. How transcription factors find their targets to initiate specific genetic programs is one of the major outstanding questions in molecular biology
%VII. IDRs are associated with phase separation
%    A. Describe phase separation in general
%    B. Describe phase separation as related to proteins
%    C. Describe functions of phase separation and relationship to subcellular structures
%VIII. IDR-mediated phase separation is a possible mechanism for initiating transcription
%    A. Summarize Stadler's papers and others
%
%AIMS
%I. Structure-function relationships of disordered proteins are unclear
%    A. Examples of studies that have found specific cases of conservation in IDRs
%    B. Alan Moses paper as reference in yeast
%II. Use evolutionary analyses to identify conserved properties of disordered regions
%    A. Conservation is powerful signal of function
%        1. Examples of other studies that applied evolution
%    B. Outline overall approach and make correspondence to chapters
%        1. Identification of orthologous groups
%        2. Application of evolutionary analyses
%
%======= DRAFT =======
Life is a physical phenomenon. Despite the complexity of living things, their processes are governed by the same physical laws that describe planets' motion around the sun and the decay of uranium atoms into lighter elements. However, many systems are too complex to describe with physical models and equations, so scientists simplify them into levels of abstraction that are more useful. (This is the origin of the common observation that biology is applied chemistry and chemistry is applied physics.) For example, Punnett squares facilitate the prediction of genotypes and phenotypes by distilling the complexities and nuances of diverse reproductive systems into a set of simple rules. However, since life spans a scale from single cells to entire ecosystems, biology likely employs more layers of abstraction than any other scientific discipline. One of the most powerful and widely used frameworks within the life sciences is biochemistry, which characterizes biological processes in terms of their component molecules and chemical reactions. A specific focus is four classes of macromolecules called nucleic acids, carbohydrates, lipids, and proteins, all of which are unique to biological systems. Though not all biological molecules are macromolecules\footnote{Macromolecule is a loose term applied to molecules with high molecular mass. In practice, it usually refers to one of the classes listed above, among a few other prominent non-biological examples.} and not all biological macromolecules fit neatly in one of these four categories, much of life at the molecular level is understood in terms of their structure and function. Each of the four has a characteristic role. Nucleic acids, \textit{i.e.} DNA and RNA, are responsible for information storage and transfer. Carbohydrates primarily store energy but can also act as structural components of cells. Lipids are a diverse class of oily molecules which are components of cell membranes, store energy, and transmit signals. Proteins have a range of functions, including catalyzing reactions, transmitting signals, transporting materials, and providing structure. Many of these overlap with the functions of the other macromolecule classes because proteins are involved in virtually every biological process. However, unlike the others, which are often passive participants, proteins are highly active and dynamic. They respond to signals, change shape, and often use the other macromolecules as substrates in their activities. Proteins are essentially the molecular machines that carry out life's functions.

Some examples will illustrate the central role of proteins more clearly. Blood is a part of the circulatory system which is responsible for transporting nutrients and waste. Though blood is a complex mixture, containing a cocktail of cells, proteins, sugars, gases, and ions dissolved in a medium of water, its primary cellular component is red blood cells. These cells, which give blood its red color, ferry oxygen from lungs throughout the body. While water can dissolve some oxygen, the body requires more oxygen more quickly than is available in the aqueous component of blood alone. Thus, red blood are packed with a special protein called hemoglobin, which binds\footnote{In biology, ``binding'' is a slippery word whose exact meaning can varying greatly depending on the context. However, it generally means that a molecule physically interacts with another molecular for an extended period.} oxygen. Each red blood cell contains as many as 270 million molecules of hemoglobin, each of which can carry up to four oxygen molecules~\cite{Pierig2008}. To put this value in perspective, if each molecule of hemoglobin were the size of a tennis ball, the hemoglobin molecules in a single red blood cell would cover a tennis court in over 10 packed layers\footnote{A standard tennis ball is 1.35 in or 0.1125 ft in radius, so the volume of 270 million tennis balls of this size packed as efficiently as possible is given by $2.70 \times 10^6 \times \frac{4}{3} \pi (0.1125 \text{ ft})^3 \times \frac{3\sqrt{2}}{\pi} = 2.17 \times 10^4 \text{ ft}^3$. A singles tennis court is 78 ft by 27 ft or 2106 ft\textsuperscript{2}, so dividing the first by the second yields slightly over 10 layers.}. Because red blood cells are so dense with hemoglobin, any defect in hemoglobin can dramatically impact the structure of the red blood cells themselves. A well-studied example is sickle cell disease where an error in the body's hemoglobin molecules deforms red blood cells into a characteristic sickle shape. This prevents them from easily flowing through blood vessels, resulting in pain and oxygen deprivation.

Whereas hemoglobin is an example of a protein mediating transport, proteins are also involved in transmitting signals and catalyzing chemical reactions. For example, the back of the eye contains a light-sensitive surface called the retina which is composed of photoreceptor cells. These cells respond to light because they produce special proteins called opsins that translate light into chemical and electrical signals which are then interpreted by the brain. Humans, and primates broadly, have three types of opsins that mediate our color vision, which are sensitive to red, green, and blue light, respectively. Color blindness is the result of photoreceptor cells missing one of these proteins, typically either the red or green opsin. In contrast to sickle cell disease, this condition is caused by a missing rather than a mutated protein. However, in other cases, a protein is not missing or mutated, but instead not produced at the right time and place. For example, lactose is a sugar found in milk that requires a specific protein, lactase, to metabolize properly. Many humans who can digest milk products in childhood, lose this ability in adulthood because they stop producing lactose. As a result, lactose in dairy products passes undigested into the colon where it is broken down by bacteria, causing symptoms such as bloating and diarrhea.

Despite performing this diverse range of functions, all proteins are made from of a set of 20 simple building blocks called amino acids.\footnote{The term amino acid encompasses any compound that contains an amino and carboxyl group. However, proteins are only synthesized from the 20 ``canonical'' amino acids. Another two (selenocysteine and pyrrolysine) are incorporated via a distinct mechanism under rare circumstances and are therefore considered non-standard.} Though each amino acid is chemically unique, they share a common backbone composed of two distinct and complementary receptor and donor sites for chemical bonds. Thus, in a protein the amino acids are bonded in a linear chain like beads on a string. However, once synthesized, proteins are not tidy rod-shaped molecules. Instead, the chain loops and weaves between itself creating a three-dimensional structure in a process called folding. These structures, which are highly stable and characteristic of each protein, are a result of the interactions between the amino acids in the chain and the surrounding medium, which is typically water. Because each amino acid has unique geometric and chemical properties that influence the energetics of these interactions, a protein's three-dimensional structure is encoded by the sequence of amino acids that compose it. A protein's function is in turn a direct result of its structure. For example, the structure of hemoglobin precisely positions its amino acids and a helper molecule called a heme group to create a pocket that can stably but reversibly bind oxygen. This allows hemoglobin to carry oxygen throughout the body until it is delivered to its destination. However, people affected by sickle cell disease have a mutation in the sequence of their hemoglobin proteins which causes it to malfunction. Frequently this mutation is a single change where the sixth amino acid in the sequence, glutamate, is substituted for a valine. This creates a sticky patch on the surface of hemoglobin, and under low-oxygen conditions normal hemoglobin changes shape to expose a sticky patch on its surface as well. The two patches are complementary, which allows hemoglobin proteins to clump together into long, fibrous strands. These strands distort the shape of red cells, giving them their characteristic sickle shape.

Clearly, understanding the relationship between the sequence, structure, and function of proteins is essential for unraveling more complex biological phenomena. Though biologists study all three properties of proteins, they are generally most interested in function since it is the most directly related to the biological processes the protein takes part in\footnote{Function generally refers to \textit{molecular} function which is a description of a specific chemical activity possessed by a protein. A biological process, however, is the larger ``biological program'' which is accomplished by the action of multiple linked molecular functions. For example, the molecular function of hemoglobin is to bind oxygen, an activity it shares with a related protein myoglobin. However, the two have different roles in the process of oxygen transport and storage. Hemoglobin is found in red blood cells where it acts as a carrier during transport. In contrast, myoglobin is found in muscle cells, where it stores oxygen until needed.}, function and biological process are not always easily identified or measured. Thus, determining a protein's structure is frequently the first step of detailed studies of its function. Though in recent years researchers have developed powerful computational tools that can accurately predict structure from sequence alone, historically structures were determined experimentally, and experimental methods still remain the gold standard. While many methods can reveal information about the structure of a protein, the most powerful techniques, X-ray crystallography, NMR spectroscopy, and cryogenic electron microscopy (cryo-EM), can map the spatial coordinates of every atom in a protein. However, this resolution requires extremely pure samples of the protein of interest. Since proteins are only produced by living systems, preparations begin with a complex mixture consisting of cells or tissue, and the protein of interest is isolated with series of extraction and purification steps. Some proteins are only produced in small amounts or degrade easily, so each step may require substantial optimization to achieve a sufficient yield. When structural techniques were first developed in the late 1950s, they were so time-consuming that a graduate student could dedicate an entire PhD to solving a single protein structure. Many developments have substantially accelerated the process, but it remains a labor-intensive technique which may require several months of effort. However, the result is a powerful map that scientists use to suggest hypotheses and interpret data.

The success of structural methods at elucidating the molecular details of protein function cemented the view that function largely depends on the presence of a fixed structure. While scientists understood proteins were not completely rigid and could adopt a variety of related structures, many believed that functional proteins largely had a single dominant structure. Despite its strength, exceptions to the structure-function paradigm were known. For example, elastin is a protein secreted by cells which allows tissues like skin or blood vessels to repeatedly expand and contract. It imparts this elasticity by forming networks of disordered chains that act like molecular springs. When a tissue experiences a force, the chains stretch to accommodate it. When the force is removed, the chains return to their random orientations, which reduces their end-to-end length and forces the tissue to return to its original shape~\cite{Vrhovski1998, Alberts2014}. However, as a result of this unique role in providing tissue elasticity, elastin's disorder was viewed as a specific adaptation rather than a general mechanism of protein function. In other cases, proteins had regions which returned undefined or highly variable atomic coordinates when analysed with structural techniques, but because these regions were often short loops between structured regions, they were seen as ``linkers'' which facilitated the structure of the functional portions of the protein. By the early 2000s, though, enough exceptions had accumulated that scientists began to recognize that fully and partially disordered proteins were involved in many biological processes~\cite{Plaxco1997, Wright1999, Dunker2001}. Many of these examples were proteins which folded on binding to their targets, commonly other proteins. This mechanism was a departure from the prevailing model of interactions between biological macromolecules which required highly stable and complementary interfaces, like two puzzle pieces fitting together. As a result, scientists speculated that disorder was an adaptation that allowed proteins to efficiently relay and regulate signals by enabling interactions with many possible targets. Furthermore, the flexibility of disordered protein would permit environmental conditions to easily modulate these interactions. In the following years, as the complete genomes of several scientifically important model organisms such as \textit{S. cerevisiae} (baker's yeast), \textit{C. elegans} (roundworm), and \textit{D. melanogaster} (fruit fly) were sequenced for the first time, researchers applied computational methods for predicting disorder to the proteins inferred from their genomes. They discovered that disorder is ubiquitous in eukaryotic\footnote{All life belongs to one of three categories, or domains. Two, Archaea and Bacteria, are all single-celled organisms with simple cellular structures. In contrast, the cells of members of Eukarya, \textit{i.e.} eukaryotes, are complex and contain substructures called organelles, among other differences. All animals, plants, and fungi are eukaryotes, though the domain includes many microorganisms as well.} organisms, with estimates of the fraction of proteins containing disordered segments of greater than 30 residues\footnote{The amino acids that compose the links of a protein chain are conventionally called residues to distinguish them from their related, but chemically distinct, free forms.} ranging between 28 and 63\%~\cite{Dunker2000, Ward2004}. For reference, though the lengths of proteins can vary dramatically, a ``typical'' protein contains on the order of a few hundred residues, so these regions can compose a significant fraction of a protein's length. Because these segments were disordered in their native state, \textit{i.e.} under normal operating conditions, such segments were termed ``intrinsically disordered regions'' (IDRs) to emphasize the disorder was not induced by exposure to chemicals or heat. Furthermore, while most proteins were predicted to contain a mixture of structured domains and IDRs, some ``intrinsically disordered proteins'' (IDPs) were entirely or almost entirely disordered. Thus, disorder was recognized as a pervasive but poorly understood feature of proteins.

Many studies investigated various structural and functional aspects of IDRs and IDPs followed in the following years.